\newpage
\section{Locks test cases}

The lock api is defined in the LFS specification: \url{https://github.com/git-lfs/git-lfs/blob/main/docs/api/locking.md}

Quote from the proposal (see \url{https://github.com/git-lfs/git-lfs/blob/main/docs/proposals/locking.md})

\begin{quote}
    \begin{enumerate}
        \item File starts out read-only locally
        \item User tries to lock a file. This is only allowed if:
        \begin{itemize}
            \item The file is not already locked by anyone else, AND
            \item One of the following are true:
            \begin{itemize}
                \item The user has, or agrees to check out, a descendant of the latest commit that was made for that file, whatever branch that was on, OR
                \item The user stays on their current commit but resets the locked file to the state of the latest commit (making it modified locally, and also cherry-picking changes for that file in practice).
            \end{itemize}
        \end{itemize}
        \item User edits file \& commits 1 or more times, on any branch they like
        \item User pushes the commits
        \item File is unlocked if:
        \begin{itemize}
            \item the latest commit to that file has been pushed (on any branch), and
            \item the file is not locally edited
        \end{itemize}
    \end{enumerate}
    
\end{quote}

\subsection{Test case 10: Locking, unlocking and listing locks sending manual requests}

\subsubsection{Description}

The client should be able to lock, unlock and list locks on a file, sending HTTP requests to the lfs server, with the correct proof of access (obtained using the "git-lfs-authenticate" command).

The client should not be able to lock a file already locked by another user.

\subsection{Test case 12: Multiple users locking and unlocking a file}

The client A lock the file. The client B should not be able to lock the file. He should have to force the unlock to delete it. 

The locks of A and B shall be listed separately when listing the locks for verification. 

A shall be able to unlock the file, and B shall need to specify "force" to unlock the file.

\subsection{Test case 13: 1 user locking and unlocking a file}

\subsubsection{Description}

Clients A commit files. If enabled, the push shall be blocked if the edited files are not locked. 

\begin{itemize}
    \item Files *.bin are marked lockable
    \item Client A lock the file a.bin
    \item \textit{The locks shall include the lock of A}
    \item Client A unlock the file a.bin
    \item \textit{The locks shall not include the lock of A}
    \item Client A lock the file a.bin
    \item \textit{The locks shall include the lock of A} 
    \item Client A push, it shall succeed
\end{itemize}

\subsection{Test case 14: 2 users fighting for a lock}

\subsubsection{Description}

Clients A and B commit files. If enabled, the push shall be blocked if the edited files are locked by another user. But when forcing, the push shall succeed.

\begin{itemize}
    \item Files *.bin are marked lockable
    \item Client A lock the file a.bin
    \item \textit{The locks shall include the lock of A}
    \item Client B attempt to lock the file a.bin. It shall fail
    \item Client B edit the file, commit it, and push it, the push shall fail.
    \item Client B try to break the lock. It shall fail
    \item Client B force the unlock. It shall succeed
    \item \textit{The locks shall not include the lock of A}
    \item Client B lock the file a.bin
    \item \textit{The locks shall include the lock of B}
    \item Client B push, it shall succeed
\end{itemize}




