\section{Lifecycle derived requirements}

In this part, we will go further by studying the lifecycle of the system. Each step in the live of the product might lead to new questions, new functions, specifications or use cases.

\subsection{Conception}

As a software system, we will most certainly base part of the system on other open-source project. This lead to 2 new requirements: 

\begin{itemize}
    \item \textbf{R12} The system shall use compatible licences, and be loosely coupled enough so the licences of components do not contaminate each other. 
    \item \textbf{R13} The system shall not use deprecated components.
\end{itemize}

\subsection{Deployement}

To minimize costs of the release and deployment cycles, the system shall meet the following requirement:

\begin{itemize}
    \item \textbf{R14} The system shall be deployable in conteneurised linux environments in a reproducible and documented way.  
\end{itemize}

\subsection{Testing}

The \textbf{R14} will also help with testing the system. The system shall be easy to deploy in several instances, as guaranteed by \textbf{R14}

\subsection{Usage}

During usage, performances are key. Each described collaboration takes a few messages in both directions, so time is definitively a concern. To address it, the system shall observe a few more requirements. 

\begin{itemize}
    \item \textbf{R15} Abstraction done of the network layer between the client and the system, the system shall handle an initial push of a 100MB of LFS as 100 large files in less than one minute.  
\end{itemize}

In the opposite direction, the system shall apply limits on the resource used. This will however be dependant on the scale of the deployment, and should then be fine-tuneable

\begin{itemize}
    \item \textbf{R16} The administrator of the system shall be able to define resource limits for the bandwidth, the storage, and the number of requests, separated between the target points (lfs, git), and discriminated by user and repository.
\end{itemize}

This introduce a new actor, the administrator of the system, that is not necessarily the administrator of the repositories and users. 

\subsection{Maintenance}

Many situations can make maintenance a nightmare, but from my experience, a few techniques can radically make things easier: guarantee memory safety, functional correctness, readable code, documentation. 

Therefore, following requirements should be added to the list: 

\begin{itemize}
    \item \textbf{R17} The system shall ensure memory safety and functional correctness either by being heavily maintained by the community, by formal technics, or by automated testing. 
    \item \textbf{R18} Every custom code shall adhere to a common and well-recognized good practices handbook, to be chosen. 
    \item \textbf{R19} The system shall be documented focussing on: the system in its whole, the system components, maintenance processes. 
\end{itemize}
